%% Generated by Sphinx.
\def\sphinxdocclass{report}
\documentclass[letterpaper,10pt,english]{sphinxmanual}
\ifdefined\pdfpxdimen
   \let\sphinxpxdimen\pdfpxdimen\else\newdimen\sphinxpxdimen
\fi \sphinxpxdimen=.75bp\relax

\PassOptionsToPackage{warn}{textcomp}
\usepackage[utf8]{inputenc}
\ifdefined\DeclareUnicodeCharacter
% support both utf8 and utf8x syntaxes
  \ifdefined\DeclareUnicodeCharacterAsOptional
    \def\sphinxDUC#1{\DeclareUnicodeCharacter{"#1}}
  \else
    \let\sphinxDUC\DeclareUnicodeCharacter
  \fi
  \sphinxDUC{00A0}{\nobreakspace}
  \sphinxDUC{2500}{\sphinxunichar{2500}}
  \sphinxDUC{2502}{\sphinxunichar{2502}}
  \sphinxDUC{2514}{\sphinxunichar{2514}}
  \sphinxDUC{251C}{\sphinxunichar{251C}}
  \sphinxDUC{2572}{\textbackslash}
\fi
\usepackage{cmap}
\usepackage[T1]{fontenc}
\usepackage{amsmath,amssymb,amstext}
\usepackage{babel}



\usepackage{times}
\expandafter\ifx\csname T@LGR\endcsname\relax
\else
% LGR was declared as font encoding
  \substitutefont{LGR}{\rmdefault}{cmr}
  \substitutefont{LGR}{\sfdefault}{cmss}
  \substitutefont{LGR}{\ttdefault}{cmtt}
\fi
\expandafter\ifx\csname T@X2\endcsname\relax
  \expandafter\ifx\csname T@T2A\endcsname\relax
  \else
  % T2A was declared as font encoding
    \substitutefont{T2A}{\rmdefault}{cmr}
    \substitutefont{T2A}{\sfdefault}{cmss}
    \substitutefont{T2A}{\ttdefault}{cmtt}
  \fi
\else
% X2 was declared as font encoding
  \substitutefont{X2}{\rmdefault}{cmr}
  \substitutefont{X2}{\sfdefault}{cmss}
  \substitutefont{X2}{\ttdefault}{cmtt}
\fi


\usepackage[Bjarne]{fncychap}
\usepackage{sphinx}

\fvset{fontsize=\small}
\usepackage{geometry}


% Include hyperref last.
\usepackage{hyperref}
% Fix anchor placement for figures with captions.
\usepackage{hypcap}% it must be loaded after hyperref.
% Set up styles of URL: it should be placed after hyperref.
\urlstyle{same}

\addto\captionsenglish{\renewcommand{\contentsname}{Contents:}}

\usepackage{sphinxmessages}
\setcounter{tocdepth}{1}



\title{Sketch}
\date{Apr 01, 2021}
\release{0.14.2}
\author{Daniel Baker}
\newcommand{\sphinxlogo}{\vbox{}}
\renewcommand{\releasename}{Release}
\makeindex
\begin{document}

\pagestyle{empty}
\sphinxmaketitle
\pagestyle{plain}
\sphinxtableofcontents
\pagestyle{normal}
\phantomsection\label{\detokenize{index::doc}}



\chapter{Features}
\label{\detokenize{index:features}}\begin{enumerate}
\sphinxsetlistlabels{\arabic}{enumi}{enumii}{}{.}%
\item {} 
\sphinxAtStartPar
Bloom Filter

\item {} 
\sphinxAtStartPar
HyperLogLog

\item {} 
\sphinxAtStartPar
SetSketch

\item {} 
\sphinxAtStartPar
Fast Hamming space distance functions

\item {} 
\sphinxAtStartPar
ngram hashing code

\end{enumerate}


\chapter{Modules}
\label{\detokenize{index:modules}}\begin{description}
\item[{There are separate modules for each sketch structure for which there are bindings.}] \leavevmode\begin{itemize}
\item {} 
\sphinxAtStartPar
sketch.hll, providing HyperLogLog and comparison, and serialization functions

\item {} 
\sphinxAtStartPar
sketch.bf, providing Bloom Filters and comparison, and serialization functions

\item {} 
\sphinxAtStartPar
sketch.bbmh, providing b\sphinxhyphen{}bit minhash implementation + comparison, and serialization functions

\item {} 
\sphinxAtStartPar
sketch.setsketch, providing set sketch + comparison, and serialization functions

\end{itemize}

\end{description}

\sphinxAtStartPar
For each of these, the module provides construction \sphinxhyphen{} either taking parameters or a path to a file.
Each of these can be written to and read from a file with .write() and a constructor.
They can be compared with each other with member functions, or you can calculate comparison matrices via
\sphinxtitleref{sketch.util.jaccard\_matrix}, \sphinxtitleref{sketch.util.containment\_matrix}, \sphinxtitleref{sketch.util.union\_size\_matrix}, \sphinxtitleref{sketch.util.intersection\_matrix}, all of which are in the util module.

\sphinxAtStartPar
Additionally, there are utilities for pairwise distance calculation in the \sphinxtitleref{util} module.


\chapter{Additional utilities: sketch.util}
\label{\detokenize{index:additional-utilities-sketch-util}}\begin{itemize}
\item {} \begin{description}
\item[{fastdiv/fastmod:}] \leavevmode\begin{itemize}
\item {} 
\sphinxAtStartPar
Python bindings for fastdiv/fastmod; See \sphinxurl{https://arxiv.org/abs/1902.01961}

\item {} 
\sphinxAtStartPar
fastdiv\_ and fastmod\_ are in\sphinxhyphen{}place modifications, while the un\sphinxhyphen{}suffixed returns a new array

\end{itemize}

\end{description}

\item {} \begin{description}
\item[{count\_eq}] \leavevmode
\sphinxAtStartPar
** Compute \# of equal registers between two 1\sphinxhyphen{}d numpy arrays.

\end{description}

\item {} \begin{description}
\item[{pcount\_eq}] \leavevmode
\sphinxAtStartPar
** Compute row\sphinxhyphen{}pair\sphinxhyphen{}wise equal register counts between two 2\sphinxhyphen{}d numpy arrays.

\end{description}

\item {} \begin{description}
\item[{shsisz}] \leavevmode
\sphinxAtStartPar
** Computes intersection size between two sorted hash sets.

\end{description}

\item {} \begin{description}
\item[{hash}] \leavevmode
\sphinxAtStartPar
** hashes strings

\end{description}

\end{itemize}


\chapter{Python\sphinxhyphen{}only Code}
\label{\detokenize{index:python-only-code}}
\begin{sphinxVerbatim}[commandchars=\\\{\},numbers=left,firstnumber=1,stepnumber=1]
\PYG{k}{def} \PYG{n+nf}{optimal\PYGZus{}ab}\PYG{p}{(}\PYG{n}{maxv}\PYG{p}{,} \PYG{n}{minv}\PYG{p}{,} \PYG{o}{*}\PYG{p}{,} \PYG{n}{q}\PYG{p}{)}\PYG{p}{:}
    \PYG{l+s+sd}{\PYGZsq{}\PYGZsq{}\PYGZsq{}}
\PYG{l+s+sd}{        Calculate a and b for maxv and minv, such that the maxv is mapped to}
\PYG{l+s+sd}{        0 and minv\PYGZsq{}s value is mapped to q.}
\PYG{l+s+sd}{        :param maxv: float value which is the maximum to be quantized}
\PYG{l+s+sd}{        :param minv: float value which is the minimum to be quantized}
\PYG{l+s+sd}{        :param q:    float or integral value for the ceiling; required.}
\PYG{l+s+sd}{        :return: namedtuple SetSketchParams, consisting of (a, b); access through ssp.a, ssp[0], or tuple access}
\PYG{l+s+sd}{    \PYGZsq{}\PYGZsq{}\PYGZsq{}}

    \PYG{k}{if} \PYG{n}{maxv} \PYG{o}{\PYGZlt{}} \PYG{n}{minv}\PYG{p}{:}
        \PYG{n}{minv}\PYG{p}{,} \PYG{n}{maxv} \PYG{o}{=} \PYG{n}{maxv}\PYG{p}{,} \PYG{n}{minv}
    \PYG{k+kn}{from} \PYG{n+nn}{numpy} \PYG{k+kn}{import} \PYG{n}{exp} \PYG{k}{as} \PYG{n}{nexp}\PYG{p}{,} \PYG{n}{log} \PYG{k}{as} \PYG{n}{nlog}
    \PYG{n}{b} \PYG{o}{=} \PYG{n}{nexp}\PYG{p}{(}\PYG{n}{nlog}\PYG{p}{(}\PYG{n}{maxv} \PYG{o}{/} \PYG{n}{minv}\PYG{p}{)} \PYG{o}{/} \PYG{n}{q}\PYG{p}{)}
    \PYG{k}{return} \PYG{n}{SetSketchParams}\PYG{p}{(}\PYG{n}{b}\PYG{o}{=}\PYG{n}{b}\PYG{p}{,} \PYG{n}{a}\PYG{o}{=}\PYG{n}{maxv} \PYG{o}{/} \PYG{n}{b}\PYG{p}{)}
\end{sphinxVerbatim}


\chapter{Indices and tables}
\label{\detokenize{index:indices-and-tables}}\begin{itemize}
\item {} 
\sphinxAtStartPar
\DUrole{xref,std,std-ref}{genindex}

\item {} 
\sphinxAtStartPar
\DUrole{xref,std,std-ref}{modindex}

\item {} 
\sphinxAtStartPar
\DUrole{xref,std,std-ref}{search}

\end{itemize}



\renewcommand{\indexname}{Index}
\printindex
\end{document}